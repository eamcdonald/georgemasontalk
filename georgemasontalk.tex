\documentclass{beamer}



%\usepackage{beamerthemesplit}
\usetheme{Boadilla}
%\usetheme{default}
%\useinnertheme{rounded}

%\useoutertheme{shadow}
\usecolortheme{rose}
%\usefonttheme{serif}
\setbeamertemplate{navigation symbols}{}
\usetheme{Madrid}

\usepackage{amssymb,amsmath,amscd,amsfonts,amsthm,dsfont,color,graphicx}
\usepackage{amscd}
%\usepackage[numbers]{natbib}
% \usepackage[french]{babel}
%\usepackage[active]{srcltx}


\def\qd{\,{\mathchar'26\mkern-12mu d}}

% \date[]{}

 \newcommand\makebeamertitle{\frame{\maketitle}}%

 \AtBeginDocument{
   \let\origtableofcontents=\tableofcontents
   \def\tableofcontents{\@ifnextchar[{\origtableofcontents}{\gobbletableofcontents}}
   \def\gobbletableofcontents#1{\origtableofcontents}
 }
\numberwithin{equation}{section}
  \theoremstyle{plain}
  \newtheorem*{thm*}{\protect\theoremname}
  \theoremstyle{plain}
  \newtheorem*{cor*}{\protect\corollaryname}
 \theoremstyle{definition}
 \newtheorem*{defn*}{\protect\definitionname}
 \theoremstyle{plain}
\newtheorem*{lem*}{\protect\lemmaname}
  \theoremstyle{plain}
  \newtheorem*{rem*}{\protect\remarkname}
   \theoremstyle{definition}
 \newtheorem*{prop*}{\protect\propositionname}

\usetheme{Madrid}

\makeatother

  \providecommand{\corollaryname}{Corollary}
  \providecommand{\definitionname}{Definitioninition}
  \providecommand{\theoremname}{Theorem}
   \providecommand{\lemmaname}{Lemma}
   \providecommand{\remarkname}{Remark}
   \providecommand{\propositionname}{Proposition}
   
   
\newcommand{\Rl}{\mathbb{R}}
\newcommand{\Cplx}{\mathbb{C}}
\newcommand{\Itgr}{\mathbb{Z}}
\newcommand{\Ntrl}{\mathbb{N}}
\newcommand{\Circ}{\mathbb{T}}
\newcommand{\Sb}{\mathbb{S}}
\newcommand{\Disc}{\mathbb{D}}
\newcommand{\Aff}{\mathbb{A}}

% The Caligraphic alphabet
\newcommand{\Ac}{\mathcal{A}}
\newcommand{\Bc}{\mathcal{B}}
\newcommand{\Cc}{\mathcal{C}}
\newcommand{\Dc}{\mathcal{D}}
\newcommand{\Ec}{\mathcal{E}}
\newcommand{\Fc}{\mathcal{F}}
\newcommand{\Gc}{\mathcal{G}}
\newcommand{\Hc}{\mathcal{H}}
\newcommand{\Ic}{\mathcal{I}}
\newcommand{\Jc}{\mathcal{J}}
\newcommand{\Kc}{\mathcal{K}}
\newcommand{\Lc}{\mathcal{L}}
\newcommand{\Mv}{\mathcal{M}}
\newcommand{\Nv}{\mathcal{N}}
\newcommand{\Oc}{\mathcal{O}}
\newcommand{\Pc}{\mathcal{P}}
\newcommand{\Qc}{\mathcal{Q}}
\newcommand{\Rc}{\mathcal{R}}
\newcommand{\Sc}{\mathcal{S}}
\newcommand{\Tc}{\mathcal{T}}
\newcommand{\Uc}{\mathcal{U}}
\newcommand{\Vc}{\mathcal{V}}
\newcommand{\Wc}{\mathcal{W}}
\newcommand{\Xc}{\mathcal{X}}
\newcommand{\Yc}{\mathcal{Y}}
\newcommand{\Zc}{\mathcal{Z}}


\newcommand{\Sp}{\mathrm{Sp}}
\newcommand{\tr}{\mathrm{tr}}
\newcommand{\Op}{\mathrm{Op}}
\newcommand{\sym}{\mathrm{sym}}
\newcommand{\Vol}{\mathrm{Vol}}
\newcommand{\Tr}{\mathrm{Tr}}
\newcommand{\dist}{\mathrm{dist}}
\newcommand{\sgn}{\operatorname{sgn}}
\newcommand{\diag}{\mathrm{diag}}
\newcommand{\id}{\mathrm{id}}
\newcommand{\Poly}{\mathrm{Poly}}

\newcommand{\spec}{\mathrm{Spec}}
\newcommand{\abs}{\mathrm{abs}}

\newcommand{\CV}{\mathrm{CV}}
\newcommand{\PCV}{\mathrm{PCV}}


% Used for highlighting. To remove all highlighting just make the command blank
\newcommand{\hl}{\color{red}}



\newcommand{\dom}{\mathrm{dom}}
\newcommand{\Bl}{\mathbb{B}^4}
\newcommand{\supp}{\mathrm{supp}}
\newcommand{\BS}{\mathfrak{BS}}
\newcommand{\dyad}{\mathrm{dyad}}
\newcommand{\Qs}{\mathscr{Q}}
\newcommand{\Av}{\mathrm{Av}}
\newcommand{\loc}{\mathrm{loc}}
% DOI transformer
\newcommand{\Ti}{\mathcal{T}}
\newcommand{\sa}{\mathrm{sa}}

\newcommand{\mf}{\mathfrak{m}}


\newcommand{\Str}{\operatorname{Str}}
   
\begin{document}

\title[Lipschitz estimates for operators]{Lipschitz estimates in Operator ideals}


\author[E. McDonald]{Edward McDonald (Penn State University)}


\institute[]{\tiny{George Mason University, January 2025} }



\makebeamertitle


\begin{frame}{Introduction}
    This talk is mostly inspired by the paper
    \begin{center}
    M., Sukochev, Lipschitz estimates in quasi-Banach Schatten ideals.
    \emph{Math. Ann.} 383 (2022), no.1--2, 571--619.
    \end{center}
\end{frame}

\begin{frame}{Plan for this talk}
    \begin{enumerate}
        \item{} Operator Lipschitz functions: some basic concepts and history.
        \item{} Schur multipliers and the L\"owner identity
        \item{} Some very light background on Schatten ideals
        \item{} Approximation methods (Besov spaces and wavelets)
    \end{enumerate}
\end{frame}


\section{Operator Lipschitz functions: some basic concepts and history}\label{section_intro}

\begin{frame}
    \Huge{Section \ref{section_intro}: Operator Lipschitz functions (Some basic concepts and history)}
\end{frame}


\begin{frame}{Basic setting}
    This talk will be about operator theory and functional calculus, but there's no real loss of generality by considering $n\times n$ matrices.
\pause
\\
    \textbf{Functional calculus for matrices:} Given a self-adjoint matrix $A \in M_{n}(\Cplx)$ and a function $f:\Rl\to \Cplx,$ the $n\times n$ matrix $f(A)$ can be defined by
    \[
        f(A) = Uf(\Lambda)U^*
    \]
    where $A = U\Lambda U^*$ is a diagonalisation, and
    \[
        f\begin{pmatrix} \lambda_1 & \cdots & 0 \\
                            \vdots & \ddots & \vdots \\
                             0      & \cdots & \lambda_n
         \end{pmatrix}
         =\begin{pmatrix} f(\lambda_1) & \cdots & 0 \\
                             \vdots      & \ddots & \vdots \\
                             0      & \cdots & f(\lambda_n)
         \end{pmatrix}.
    \]
\end{frame}

\begin{frame}{Operator Lipschitz functions}
    Recall that the operator norm $\|\cdot\|_{\infty}$ of a matrix $A \in M_n(\Cplx)$ may be defined as
    \[
        \|A\|_{\infty} := \sup_{|x|_2\leq 1} |Ax|_2.
    \]
    where $|\cdot|_2$ is the Euclidean norm.\\
    \pause
    A function $f:\Rl\to \Cplx$ is called \emph{Lipschitz} if there exists a constant $c_f$ such that
    \[
        |f(t)-f(s)|\leq c_f|t-s|,\quad t,s\in \Rl.
    \]\pause
    Similarly, a function $f:\Rl\to \Cplx$ is said to be \emph{operator Lipschitz} if there exists a constant $C_f$ (independent of $n$) such that
    \begin{equation*}
        \|f(A)-f(B)\|_\infty \leq C_f\|A-B\|_\infty,\quad A=A^*,B=B^* \in M_n(\Cplx).
    \end{equation*}
    \pause
    \begin{block}{Question (from M.~G.~Krein, 1950s)}
        Is every Lipschitz function operator Lipschitz?
        \pause
        That is, does $|f(t)-f(s)|\lesssim |t-s|$ imply that $\|f(A)-f(B)\|_{\infty} \lesssim \|A-B\|_{\infty}?$
    \end{block}
\end{frame}

\begin{frame}{Operator Lipschitz functions}
    \begin{block}{Answer}
        No.\\\pause
        Farforovskaya (1968): There exist Lipschitz functions that are not operator Lipschitz\\\pause
        Kato (1973): The absolute value function $f(t) = |t|$ is not operator Lipschitz\\\pause
        Johnson \& Williams (1975): An operator Lipschitz function is differentiable.
    \end{block}
\end{frame}

\begin{frame}{Dimension-dependent constant}
    If $f$ is any Lipschitz function on $\Rl,$ and $A,B \in M_n(\Cplx)$ are self-adjoint then
    \[
        \|f(A)-f(B)\|_{\infty} \leq C_{\mathrm{abs}}\log(1+n)\|f\|_{\mathrm{Lip}}\|A-B\|_{\infty}
    \]
    where $C_{\mathrm{abs}}$ is an absolute constant. This is sharp in the order of growth as $n\to\infty.$ I do not know if a sharp estimate for $C_{\mathrm{abs}}$ is known.
\end{frame}


\begin{frame}{Operator Lipschitz functions}
    It is easy to check that sufficiently good functions are operator Lipschitz.\\
    Let's check the function $f(x) = e^{i\xi x}$ for $\xi\in \Rl.$ There's a classical integral identity:
    \begin{equation*}
        e^{i\xi A}-e^{i\xi B} = i\xi\int_{0}^1 e^{i\xi (1-\theta)A}(A-B)e^{i\xi \theta B}\,d\theta.
    \end{equation*}
    This is called the Duhamel formula, you can check it by differentiating both sides. The triangle inequality implies
    $$
        \|e^{i\xi A}-e^{i\xi B}\|_{\infty} \leq |\xi|\|A-B\|_{\infty}.
    $$
    \pause
    By Fourier inversion,
    \begin{equation*}
        \|f(A)-f(B)\|_{\infty} \leq \|A-B\|_\infty\cdot 2\pi \|\widehat{f'}\|_1.
    \end{equation*}
    \pause
    By Cauchy-Schwarz and Plancherel, $\|\widehat{f'}\|_1 \leq \|f'\|_2+\|f''\|_2.$ This is a ``good enough" sufficient condition for most purposes.
\end{frame}

\begin{frame}{Peller's theorem}
    The previous computation was based on Fourier inversion of $f$ and a description of $e^{i\xi A}-e^{i\xi B}$ as an integral (Duhamel's integral).
    \pause
    Using a more subtle description of $e^{i\xi A}-e^{i\xi B}$, and handling the Littlewood-Paley components of $f$
    individually, V. V. Peller has proved the following:
    \begin{theorem}[Peller (1990)]
        If $f$ is Lipschitz and belongs to the homogeneous Besov class $\dot{B}^1_{\infty,1}(\Rl)$ then $f$ is operator Lipschitz.
    \end{theorem}
    \pause
    In other words, if $f$ is Lipschitz and
    \begin{equation*}
        \int_0^\infty \sup_{t\in \Rl} \frac{|f(t-h)-2f(t)+f(t+h)|}{h^2} \,dh < \infty
    \end{equation*}
    then $f$ is operator Lipschitz. 
    \pause
    For example, if $f',f'' \in L_{\infty}(\Rl)$ then $f$ is operator Lipschitz.
\end{frame}

\begin{frame}{Peller's operator Bernstein inequality}
    The classical Bernstein inequality states that if $f \in L_{\infty}(\Rl)$ has Fourier transform supported in the interval $[-\sigma,\sigma],$ then
    \[
        \|f\|_{\mathrm{Lip}} \leq C\sigma \|f\|_{\infty}.
    \]\pause
    Peller's theorem is a consequence of his \emph{operator Bernstein inequality}.
    \begin{theorem}[Peller (1990)]
        If $f \in L_{\infty}(\Rl)$ has Fourier transform supported in the interval $[-\sigma,\sigma],$ then
        \[
            \|f\|_{\mathrm{O-Lip}} \leq C\sigma\|f\|_\infty.
        \]
    \end{theorem}
    Here $\|f\|_{\mathrm{O-Lip}}$ is the operator Lipschitz seminorm, i.e.
    \[
        \|f\|_{\mathrm{O-Lip}} := \sup_{A=A^*,B=B^*\in \Bc(H)} \frac{\|f(A)-f(B)\|_{\infty}}{\|A-B\|_{\infty}}.
    \]
\end{frame}

\section{Schur multipliers and the L\"owner identity}\label{section_schur}

\begin{frame}
    \Huge{Section \ref{section_schur}}
\end{frame}

\begin{frame}{Another perspective: Schur multipliers}
    Let $A,B \in M_n(\Cplx).$ The Schur product (a.k.a. Hadamard product, entrywise product...) of $A$ and $B$ is defined by
    \[
        (A_{j,k})_{j,k}\circ (B_{j,k})_{j,k} = (A_{j,k}B_{j,k})_{j,k}.
    \]
    \pause
    That is, $A\circ B$ is the matrix formed by multiplying the entries of $A$ and $B$ componentwise.\\

    What is the point of such a thing?
\end{frame}


\begin{frame}{L\"{o}wner's formula}
    The Schur product is very important in linear algebra. One reason for this is that it can be used to \emph{linearise}
    nonlinear operators.\\ \pause
    If $A$ and $B$ are self-adjoint matrices with eigendecompositions
    \[
        A\xi_j= \lambda_j(A)\xi_j,\quad B\eta_k = \lambda_k(B)\eta_k,\quad 0\leq j,k < n
    \]
    then
    \[
        f(A)\xi_j = f(\lambda_j(A))\xi_j,\quad f(B)\eta_k = f(\lambda_k(B))\eta_k,\quad 0\leq j,k < n.
    \]
    K.~L\"owner noticed that therefore
    \[
        \langle \xi_j,(f(A)-f(B))\eta_k\rangle =  \frac{f(\lambda_j(A))-f(\lambda_k(B))}{\lambda_j(A)-\lambda_k(B)}\langle \xi_j,(A-B)\eta_k\rangle
    \]
    for all $0\leq j,k < n.$
\end{frame}

\begin{frame}{L\"{o}wner's formula}
    In other words, let $\Psi_{f,A,B}$ denote the ``L\"{o}wner matrix"
    \[
        \Psi_{f,A,B} = \left(\frac{f(\lambda_j(A))-f(\lambda_k(B))}{\lambda_j(A)-\lambda_k(B)}\right)_{j,k=0}^{n-1}
    \]
    then
    \[
        f(A)-f(B) = \Psi_{f,A,B} \circ (A-B)
    \]
    where the Schur product $\circ$ is computed with respect to the matrix basis $\{\xi_j\otimes \eta_k\}_{j,k=0}^{n-1}.$\\
    \pause
    This means that studying the highly nonlinear relationship
    \[
        A-B \mapsto f(A)-f(B)
    \]
    is reduced to the study of the linear map
    \[
        X\mapsto \Psi_{f,A,B}\circ X.
    \]
\end{frame}

\begin{frame}{How to study operator Lipschitz functions}
    Therefore if we want to find the operator Lipschitz functions, we need to characterise
    those functions $f$ such that for all $A$ and $B$, we have
    \[
        \|\Psi_{f,A,B}\circ X \|_{\infty} \leq C_f\|X\|_{\infty}.
    \]
    Because this is a linear (rather than nonlinear) problem, it turns out to be much easier.
\end{frame}

\section{Some very light background on Schatten ideals}\label{section_schatten}

\begin{frame}
    \Huge{Section \ref{section_schatten}: Some very light background on Schatten ideals}
\end{frame}



\begin{frame}{Schatten ideals}
    If $T$ is a compact operator on $H$, the singular value sequence of $T$ is defined as
    $$
        \mu(k,T) := \inf\{\|T-R\|_{\infty}\;:\;\mathrm{rank}(R)\leq k\},\quad k\geq 0.
    $$
    (Equivalently, $\mu(T) = \{\mu(k,T)\}_{k=0}^\infty$ is the sequence of eigenvalues of the absolute value $|T|$ arranged in non-increasing order with multiplicities.)\\
    \pause
    Note that $\|T\|_\infty = \mu(0,T) = \|\mu(T)\|_{\ell_\infty}.$
    \pause
    For $1\leq p < \infty$, the Schatten $\Lc_p$-norm of a compact operator $T$ is
    $$
        \|T\|_p := \|\mu(T)\|_{\ell_p} = \left(\sum_{k=0}^\infty \mu(k,T)^p\right)^{\frac1p}.
    $$
    Equivalently, $\|T\|_p = \Tr(|T|^p)^{1/p}.$
    It is not obvious, but this is a norm (i.e. $\|T+S\|_p\leq \|T\|_p+\|S\|_p.$)
\end{frame}

\begin{frame}{$\Lc_p$-operator Lipschitz functions}
    A function $f$ on $\Rl$ is said to be $\Lc_p$-operator Lipschitz if there exists a constant $C_f>0$ such that
    $$
        \|f(A)-f(B)\|_p \leq C_f\|A-B\|_p,\quad A=A^*,B=B^* \in M_n(\Cplx)
    $$
    By Loewner's identity, this follows from (actually, is equivalent to)
    \[
        \|\Psi_{f,A,B}\circ X\|_p \leq C_f\|X\|_p,\quad X\in M_n(\Cplx).
    \]
\end{frame}

\begin{frame}{$\Lc_2$-operator Lipschitz function}
    By far the easiest case is $p=2,$ because the $\Lc_2$ norm of a matrix is the same as the $\ell_2$ norm of its entries
    \[
        \|\{A_{j,k}\}_{j,k}\|_2 = \left(\sum_{j,k} |A_{j,k}|^2\right)^{1/2}.
    \]
    Therefore
    \[
        \|M\circ A\|_{2} \leq \max\{|M_{j,k}|\}_{j,k} \|A\|_2.
    \]
    It follows that
    \begin{align*}
        \|f(A)-f(B)\|_2 &= \|\Psi_{f,A,B}\circ (A-B)\|_2\\
                        &\leq \max_{j,k} \left\{\left|\frac{f(\lambda_j(A))-f(\lambda_k(B))}{\lambda_j(A)-\lambda_k(B)}\right|\right\}\|A-B\|_2\\
                        &\leq\|f\|_{\mathrm{Lip}}\|A-B\|_2.
    \end{align*}
\end{frame}



\begin{frame}{What about $p\neq 2$?}
    So any Lipschitz function $f$ is Lipschitz in $\Lc_2.$ What about other Schatten ideals?
    \begin{theorem}[Potapov and Sukochev (2010)]
        For $1 < p < \infty$, all Lipschitz functions are $\Lc_p$-operator Lipschitz.
    \end{theorem}
    \pause
    For $p\neq 2$, this requires some very deep harmonic analysis. \pause
    There are now essentially four proofs of Potapov-Sukochev. The most recent is relatively simple and due to
    Conde-Alonso, Gonz\'alez-P\'erez, Parcet and Tablate, but still uses advanced operator-valued harmonic analysis.
\end{frame}



\begin{frame}{What about $0 < p < 1$?}
    For $0 < p < 1$, we can still define
    $$
        \|T\|_p := \|\mu(T)\|_{\ell_p} = \Tr(|T|^p)^{\frac1p}.
    $$
    This is not a norm, merely a quasi-norm. There is no triangle inequality, merely a quasi-triangle inequality
    $$
        \|T+S\|_p \leq 2^{\frac1p-1}(\|T\|_p+\|S\|_p).
    $$
    \pause
    Nonetheless, we have
    \[
        \|T+S\|_p^p \leq \|T\|_p^p+\|S\|_p^p.
    \]
\end{frame}

%
% \begin{frame}{Geometry in $\Lc_p.$}
%     The unit ball $B = \{T\;:\; \|T\|_p\leq 1\}$ in $\Lc_p$ is not convex.
%
%     i.e., if $\xi_1,\ldots,\xi_n\in B$ then it might happen that
%     \[
%         \theta_1\xi_1+\cdots+\theta_n\xi_n\notin B,\quad |\theta_1|+\cdots+|\theta_n|\leq 1.
%     \]
%     For this reason the theory of integration $\Lc_p$-valued functions is not straightforward. We could have continuous functions $f \in C([0,1],\Lc_p)$
%     whose integral is not in $\Lc_p.$\\
%     \pause
%     Instead, $B$ is only closed under $p$-convex combinations, i.e.
%     \[
%         \theta_1\xi_1+\cdots+\theta_n \xi_n \in B,\quad |\theta_1|^p+\cdots+|\theta_n|^p \leq 1.
%     \]
% \end{frame}
%

\begin{frame}{$\Lc_p$-Lipschitz functions for $0 < p < 1.$}
    Which functions are Lipschitz in $\Lc_p$ when $0 < p < 1$? \\
    \pause
    At least some functions are, for example $f(t) = (t+\lambda)^{-1}$, $\lambda \in \Cplx\setminus \Rl.$\\
    \pause
    What about $f(t) = \exp(it\xi)$ for $\xi\in \Rl$?
\end{frame}

% \begin{frame}{Schur multipliers}
%     The usual method to prove $\Lc_p$-estimates in $0<p<1$ is the same as for $p\geq 1,$ to use Schur multipliers.
%
%     If $A = \{A_{j,k}\}_{j,k}$ and $B = \{B_{j,k}\}_{j,k}$ are matrices of the same size, then $A\circ B := \{A_{j,k}B_{j,k}\}_{j,k}.$
%
%     \begin{definition}
%         Let $m = \{m_{j,k}\}_{j,k=1}^n$ be an $n\times n$ matrix. Define
%         \[
%             \|m\|_{\mf_p} := \sup_{\|B\|_{p}\leq 1} \|m\circ B\|_p.
%         \]
%         In general, let $m:\Rl^2\to \Cplx$ be bounded and define
%         \[
%             \|m\|_{\mf_p} := \sup_{x_1,\ldots,x_n,y_1,\ldots,y_n\in \Rl} \|\{m(x_j,y_k)\}_{j,k=1}^n\|_{\mf_p}.
%         \]
%     \end{definition}
% %     We have some obvious properties for $0<p\leq 1:$
% %     \begin{itemize}
% %         \item{} $\|m_1+m_2\|_{\mf_p}^p \leq \|m_1\|_{\mf_p}^p+\|m_2\|_{\mf_p}^p.$
% %         \item{} If $m_1$ is a principal submatrix of $m_2,$ then $\|m_1\|_{\mf_p} \leq \|m_2\|_{\mf_p}.$
% %     \end{itemize}
% %     We could also consider the ``cb" version. This does not make much difference.
% \end{frame}
%
% \begin{frame}{Schur multipliers and operator-Lipschitz functions}
%     A folk result:
%     \begin{theorem}[Hadamard(?), Schur(?), L\"owner(?), Daletskii-Krein(?)]
%         Let $1\leq p\leq \infty,$ and let $f$ be a measurable function on $\Rl.$ Define
%         \[
%             f^{[1]}(\lambda,\mu) = \begin{cases}
%                                         \frac{f(\lambda)-f(\mu)}{\lambda-\mu},\quad \lambda \neq \mu\\
%                                         f'(\lambda),\quad \lambda=\mu.
%                                    \end{cases}
%         \]
%         Then $f$ is $\Lc_p$-Lipschitz if and only if $f^{[1]}$ is a bounded Schur multiplier.
%     \end{theorem}
%     \pause
%     If $f$ is not differentiable then $f'(\lambda)$ does not make sense everywhere, but this is no big deal. Bounded diagonal matrices are Schur multipliers in $\Lc_p$
%     for any $1\leq p\leq \infty$ so we can make $f^{[1]}(\lambda,\lambda)$ any bounded function of $\lambda$ without changing the result.
% \end{frame}
%
%
%
%
% \begin{frame}{Schur multipliers in $\Lc_p$}
%     One noteworthy difference between $p=1$ and $p<1$ is the following example:
%     \begin{example}
%         Let $I_n = \{\delta_{j,k}\}_{j,k=0}^{n-1}$ be the $n\times n$ identity matrix. Then
%         \[
%             \|I_n\|_{\mf_p} = n^{\frac1p-1}.
%         \]
%         To see this, compute $I_n\circ (\xi\otimes \xi)$ where $\xi = (1,\ldots,1).$
%     \end{example}
%     \pause
%     What this means is that restriction to the diagonal
%     \[
%         \{A_{j,k}\}_{j,k\geq 0} \mapsto \{A_{j,j}\delta_{j,k}\}_{j,k=0}^\infty
%     \]
%     is not bounded in $\Lc_p$ for any $0<p<1$!
% \end{frame}
%
%
%
% \begin{frame}{Schur multipliers and operator-Lipschitz functions in $\Lc_p$ for $0<p<1.$}
%     \begin{theorem}
%         Let $0< p\leq \infty.$ and let $f$ be a measurable function on $\Rl.$ Then $f$ is $\Lc_p$-Lipschitz if and only if
%         \[
%             \sup_{\{\lambda_j\},\{\mu_j\}} \|\{f^{[1]}(\lambda_j,\mu_k)\}_{j,k}\|_{\mf_p} < \infty
%         \]
%         where the supremum is over all \emph{disjoint} sequences $\{\lambda_j\}_{j}$ and $\{\mu_k\}_k.$
%     \end{theorem}
%     Since we only consider disjoint sequences, the diagonal does not enter the picture.
% \end{frame}
%
%
% \begin{frame}{An example}
%     Consider the function
%     \[
%         f(x) = \sin(x)
%     \]
%     and $\mu_j=\lambda_j=2\pi j.$ Then
%     \[
%         f^{[1]}(\lambda_j,\mu_k) = \delta_{j,k}.
%     \]
%     But this is not a Schur multiplier of $\Lc_p$! The same reasoning applies to any periodic function $f$ with $f'(\lambda)\neq 0$ for some $\lambda.$
%     \pause
%     This argument is obviously flawed, we are supposed to consider \emph{disjoint} sequences.
% \end{frame}
%
% \begin{frame}{A corrected example}
%     We can fix the previous (wrong) argument by shifting one of the sequences by $\varepsilon>0.$ Let $f$ be a $1$-periodic function. Consider the sequences
%     \[
%         \lambda_j = j+\varepsilon, \mu_k = k,\quad j,k\geq 0.
%     \]
%     Then
%     \[
%         f^{[1]}(\lambda_j,\mu_k) = \frac{f(j+\varepsilon)-f(k)}{j-k+\varepsilon} = (f(\varepsilon)-f(0))\frac{1}{j-k+\varepsilon},\quad j,k\geq 0.
%     \]
%     If $f(\varepsilon)\neq f(0),$ then we need to consider the matrix
%     \[
%         \{(j-k+\varepsilon)^{-1}\}_{j,k\geq 0}.
%     \]
%     This matrix is not diagonal, but a straightforward modification of the argument for $\{\delta_{j,k}\}_{j,k\geq 0}$ shows that it is not a Schur multiplier of $\Lc_p$ for any $0<p<1.$
%     \pause
%     Aleksandrov and Peller have characterised Schur multipliers of $\Lc_p$ of the Herz-Toeplitz form $m(j-k),$ so we could also use their result.
% \end{frame}
%
%

\begin{frame}{Periodic functions are not $\Lc_p$-Lipschitz for $0 < p < 1$.}
    \begin{lemma}[M. and Sukochev (2022)]
        Let $0 < p < 1$, and let $f$ be a periodic function on $\Rl$. Then $f$ is $\Lc_p$-Lipschitz
        if and only if it is constant.
    \end{lemma}
    What does this imply?\pause
    \begin{itemize}
        \item{} Even $C^\infty$ functions with all derivatives bounded may not be $\Lc_p$-Lipschitz;\\
        \item{} In particular $f(t) = \exp(it\xi)$, $\xi\neq 0$ is not $\Lc_p$-Lipschitz for any $0 < p < 1.$ This means that methods based on a Fourier decomposition
        are unlikely to work.
    \end{itemize}
\end{frame}
% %
% % \begin{frame}{Fourier multipliers}
% %     An analogous issue is Fourier multipliers in $L_p(\Circ)$ for $0<p<1.$
% %     \begin{theorem}
% %         Let $m \in \ell_{\infty}(\Itgr)$ and $0<p<1.$ The Fourier multiplier
% %         \[
% %             T_m:L_2(\Circ)\to L_2(\Circ),\quad T_m(\exp(i\theta n)) = m(n)\exp(i\theta n),\quad n\in \Itgr
% %         \]
% %         is bounded on $L_p(\Circ)$ if and only if $m$ has the form
% %         \[
% %             m(n) = \sum_{j=0}^\infty c_j \exp(in\zeta_j),\quad n\in \Itgr
% %         \]
% %         where $\sum_{j} |c_j|^p < \infty.$
% %     \end{theorem}
% %     In other words, the only $L_p(\Circ)$ multipliers for $0<p<1$ are shift operators and $p$-convex combinations of shifts.
% % \end{frame}
% %
% % \begin{frame}{Strategies to get $\Lc_p$-operator Lipschitz estimates}
% %     In the $\Lc_{\infty}$ case, we started with a class of functions $\{\exp(i\xi x)\}_{\xi\in \Rl}$ for which Lipschitz estimates are easy, and derived a more general class by taking convex combinations.
% %
% %     If we could find some set $\{\psi_j\}$ of functions which we know are $\Lc_p$-Lipschitz, then we could conclude that functions of the form
% %     \[
% %         \sum_j c_j\psi_j
% %     \]
% %     are also $\Lc_p$-operator Lipschitz.
% % \end{frame}
% %
% % \begin{frame}{Strategies to get $\Lc_p$-operator Lipschitz estimates}
% %     We know that if $f_\lambda(t) = (t+\lambda)^{-1},$ where $\lambda\in \Cplx\setminus \Rl,$ then
% %     \[
% %         \|f_{\lambda}\|_{\Lc_p-\mathrm{Lip}} \leq |\Im(\lambda)|^{-2}.
% %     \]
% %     Essentially every smooth function on $\Rl$ belongs to the closed convex hull of $\{f_\lambda\}_{\Im(\lambda)\neq 0}.$\\
% %     \pause
% %     I tried for a long time to characterise functions $f$ having a decomposition like
% %     \[
% %         f(t) = \sum_{j=0}^\infty c_j|\Im(\lambda_j)|^2f_{\lambda_j}(t)
% %     \]
% %     where $\sum_{j=0}^\infty |c_j|^p < \infty,$ but with no success.
% % \end{frame}
%
% \begin{frame}{An idea}
%     Consider the matrix
%     \begin{equation}\label{model_schur}
%         \left\{\frac{c_j-c_k}{j-k+\varepsilon}\right\}_{j,k\geq 0}
%     \end{equation}
%     where $c_j$ is a scalar sequence. This is approximately a model for $f^{[1]}(j+\varepsilon,k)$ where $f$ is a function of the form
%     \[
%         f(x) = \sum_{k\in \Itgr} c_k\psi(x-k)
%     \]
%     where $\psi$ is some bump function.
%     \begin{lemma}
%         If $\sum_{j} |c_j|^{\frac{p}{1-p}}<\infty$ then \eqref{model_schur} is a Schur multiplier of $\Lc_p.$
%     \end{lemma}
%     \pause
%     Why is it $\frac{p}{1-p}$? It comes down to the inequality
%     \[
%         (\sum_{j} |a_jb_j|^p)^{\frac1p} \leq (\sum_{j} |a_j|^{\frac{p}{1-p}})^{\frac{1-p}{p}}(\sum_{j} |b_j|)
%     \]
% \end{frame}
%
% \begin{frame}{Sums of shifted bump functions}
%     With considerably more effort, it is possible to prove the following:
%     \begin{theorem}
%         Let $\psi \in C^k_c(\Rl),$ where $k>\frac{2}{p}-1.$ If $\{c_k\}_{k\in \Itgr}$ is some scalar sequence, and
%         \[
%             f(x) = \sum_{k\in \Itgr} c_k\psi(x-k)
%         \]
%         then
%         \[
%             \|f^{[1]}\|_{\mf_p} \lesssim \|\{c_k\}_{k\in \Itgr}\|_{\ell_{\frac{p}{1-p}}}.
%         \]
%     \end{theorem}
%     What kind of functions can we build out of functions like this?
% \end{frame}
%

\section{Approximation methods (Besov spaces and wavelets)}\label{section_wavelets}

\begin{frame}
    \Huge{Section \ref{section_wavelets}: Approximation methods (Besov spaces and wavelets)}
\end{frame}


\begin{frame}{Wavelet methods}
    What is a good way of approximating a general function by nicer functions?
    \begin{theorem}[Daubechies (1988)]
        For all $k>0$, there exists a compactly supported $C^k$ function $\psi$ such that the system of translations
        and dilations
        \begin{equation*}
            \psi_{j,k}(t) := 2^{\frac{j}{2}}\psi(2^jt-k),\quad j,k\in \Itgr
        \end{equation*}
        forms an orthonormal basis of $L_2(\Rl).$
    \end{theorem}
\end{frame}


\begin{frame}{Wavelet methods}
    Wavelets are analogous to Fourier series, in the sense that if
    \[
        f(t) = \sum_{j\in \Itgr} \sum_{k\in \Itgr} c_{j,k}\psi_{j,k}(t)
    \]
    then the coefficients $c_{j,k}$ for $j>N$ represent oscillations of $f$ on the scale $\sim 2^{-N}.$ A function of the form
    \[
        f(t) = \sum_{j<N} \sum_{k\in \Itgr}c_{j,k}\psi_{j,k}(t)
    \]
    does not oscillate greatly on scales smaller than $2^{-N}.$ This is similar to functions with Fourier transform supported in $[-2^N,2^N].$
\end{frame}

\begin{frame}{An $\Lc_p$-Lipschitz Bernstein inequality}
        \begin{theorem}[M.-Sukochev (2022)]
        Let $f\in L_{\frac{p}{1-p}}(\Rl)$ have Wavelet expansion
        \[
            f(t) = \sum_{j\in \Itgr} \sum_{k\in \Itgr} c_{j,k}\psi_{j,k}(t)
        \]
        where $c_{j,k}=0$ for $k>N.$ Then
        \[
            \|f^{[1]}\|_{\mf_p} \leq C2^{\frac{N}{p}} \|f\|_{\frac{p}{1-p}}.
        \]
    \end{theorem}
    With $p=1,$ this is the wavelet analogy of Peller's operator Bernstein inequality. For $p<1$ it is new.
\end{frame}


\begin{frame}{Wavelets and Besov spaces}
    It follows from the Wavelet Bernstein inequality that Besov spaces have a very simple characterisation in terms of wavelet coefficients.
    \begin{theorem}[Meyer (1986)]
        Let $s \in \Rl$ and $p,q\in (0,\infty].$ Let $\psi$ be a compactly supported $C^k$ wavelet where $k > -s.$ Then a distribution $f\in \Dc'(\Rl)$
        belongs to the homogeneous Besov space $\dot{B}^{s}_{p,q}(\Rl)$ if and only if
        \begin{equation*}
            \|f\|_{B^s_{p,q}}\approx \sum_{j\in \Itgr} 2^{jq(s+\frac{1}{2}-\frac{1}{p})}\left(\sum_{k\in \Itgr} |\langle f,\psi_{j,k}\rangle|^p\right)^{\frac{q}{p}} < \infty.
        \end{equation*}
    \end{theorem}
\end{frame}



\begin{frame}{A new result}
    Using the $p$-triangle inequality and the $\Lc_p$-Lipschitz Bernstein inequality, we get the following:
%     we easily conclude that $\|f\|_{\Lc_p-\mathrm{Lip}} \lesssim \|f\|_{B^{\frac1p}_{\frac{p}{1-p},p}(\Rl)}.$
    \begin{theorem}[M. and Sukochev (2022)]
        Let $0 < p < 1.$ Let $f \in \dot{B}^{\frac1p}_{\frac{p}{1-p},p}(\Rl)$ be Lipschitz continuous. Then $f$ is $\Lc_p$-Lipschitz and
        for all self-adjoint matrices $A$ and $B$,
        \begin{align*}
            \|f(A)-f(B)\|_p \leq C_{p}(\|f\|_{\mathrm{Lip}}+\|f\|_{\dot{B}^{\frac1p}_{\frac{p}{1-p},p}(\Rl)})\|A-B\|_p
        \end{align*}
    \end{theorem}
    \pause
    In other words, we require that $f$ be Lipschitz and for some $n>\frac1p$ that
    \begin{equation*}
        \int_0^\infty \left(\int_{-\infty}^\infty \left|\sum_{k=0}^n \binom{n}{k}(-1)^{n-k}f(t+kh)\right|^{\frac{p}{1-p}}\,dt\right)^{1-p} \frac{dh}{h^2} < \infty.
    \end{equation*}
    \pause
    For example, $f',\ldots f^{(k)} \in L_{\frac{p}{1-p}}(\Rl)$ where $k > \frac{1}{p}-1$ is sufficient.
\end{frame}

% \begin{frame}{What else can we do?}
%     Wavelets are not new, but their application to this theory is. \pause
%     Some other things we can achieve:
%     \begin{itemize}
%         \item{} For all $n\geq 0$, the inequality
%         $$
%             \sum_{k=0}^n \mu(k,f(A)-f(B))^p \lesssim (\|f'\|_\infty+\|f\|_{\dot{B}^{\frac{1}{p}}_{\frac{p}{1-p},p}})\sum_{k=0}^n \mu(k,A-B)^p
%         $$
%         (this recovers the previous result with $n=\infty.$)\pause
%         \item{} H\"older-type estimates of the form
%         $$
%             \|f(A)-f(B)\|_{p} \lesssim_f \||A-B|^{\alpha}\|_p
%         $$
%         for $f$ in some Besov space.
%     \end{itemize}
% \end{frame}
% \begin{frame}{Wavelet Bernstein inequality}
%     How do we use wavelet methods? The key is again a Bernstein inequality.
%     \begin{theorem}[Meyer(?) (1980s)]
%         Let $f\in L_{\infty}(\Rl)$ have Wavelet expansion
%         \[
%             f(t) = \sum_{j\in \Itgr} \sum_{k\in \Itgr} c_{j,k}\psi_{j,k}(t)
%         \]
%         where $c_{j,k}=0$ for $k>N.$ Then
%         \[
%             \|f\|_{\mathrm{Lip}} \leq C2^N \|f\|_{\infty}.
%         \]
%     \end{theorem}
% \end{frame}
%
%
%
%
% \begin{frame}{Future directions}
%     If $A$ and $B$ are two bounded operators, then
%     \[
%         f(A+B) = f(A)+T^{A,A}_{f^{[1]}}(B)+T^{A,A,A}_{f^{[2]}}(B,B)+\cdots+T^{A,A,\cdots,A+B}_{f^{[k+1]}}(B,B,\cdots,B).
%     \]
%     Here $T^{A,\cdots,A}_{f^{[k]}}$ are \emph{multiple operator integrals}, essentially multilinear Schur multipliers.
%     \pause
%     If $B \in \Lc_p,$ we would like to understand when the Taylor remainder
%     \[
%         f(A+B)-f(A)-T^{A,A}_{f^{[1]}}(B)-\cdots -T^{A,A,\cdots,A}_{f^{[n]}}(B,\cdots,B)
%     \]
%     also belongs to $\Lc_p.$
% \end{frame}
%
% \begin{frame}{Bilinear Schur multipliers}
%     Let $m:\Rl^3\to \Cplx$ be a function of three variables. Given finitely supported matrices $X,Y\in \ell_{\infty}(\Rl^2),$ let
%     \[
%         m\circ (X,Y)(\lambda,\nu) = \sum_{\mu} m(\lambda,\mu,\nu)X(\lambda,\mu)Y(\mu,\nu).
%     \]
%     \begin{definition}[Bilinear Schur multipliers]
%         Given $0<p_1,p_2,p_3\leq \infty,$ let
%         \[
%             \|m\|_{\mf(\Lc_{p_1}\times \Lc_{p_2},\Lc_{p_3})} = \sup_{\|A\|_{p_1}\leq 1,\, \|B\|_{p_2}\leq 1} \|m\circ (X,Y)\|_{p_3}.
%         \]
%     \end{definition}
%     Similarly we can define $n$-linear Schur multipliers for $n>2.$
% \end{frame}
%
% \begin{frame}{Multilinear Schur multipliers}
%     \begin{theorem}[Potapov-Sukochev-Skripka, Le Merdy-Skripka]
%         If
%         \[
%             1<p<\infty.
%         \]
%         and $f^{(n)} \in C_b(\Rl),$ then
%         \[
%             \|f^{[n]}\|_{\mf(\Lc_{np}^n,\Lc_p)}\leq C\|f^{(n)}\|_{\infty}.
%         \]
%     \end{theorem}
%     \begin{theorem}[Peller]
%         If $1\leq p \leq \infty,$ then
%         \[
%             \|f^{[n]}\|_{\mf(\Lc_{np}^n,\Lc_p)}\leq C\|f\|_{B^n_{\infty,1}(\Rl)}
%         \]
%     \end{theorem}
%     Question: could this be extended to $0<p<1$?
% \end{frame}
%
% \begin{frame}{Multilinear Schur multipliers}
%     The best I have so far weakens $\Lc_{np}$ to $\Lc_p.$
%     \begin{theorem}
%         Let $0 < p \leq 1.$ If $f \in B^{\frac{n}{p}}_{\frac{p}{1-p},p}(\Rl)$ then
%         $f^{[n]} \in \mf(\Lc_p^n,\Lc_p).$
%     \end{theorem}
%     This implies that if $f \in B^{n/p}_{\frac{p}{1-p},p}(\Rl),$ $A\in \Bc_{\sa}(H)$ and $X = X^* \in \Lc_p,$ then the function
%     \[
%         t\mapsto f(A+tX)
%     \]
%     admits an $n$-term Taylor expansion with remainder term in $\Lc_p$ with quasi-norm of size $O(t^n).$
% \end{frame}
%


\begin{frame}
\structure{\begin{center}
{\Huge{}Thank you for listening!}
\par\end{center}}\end{frame}



\end{document}

